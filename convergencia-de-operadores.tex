\documentclass[xcolor=dvipsnames]{beamer}
\usepackage[utf8]{inputenc}
\usepackage{amsmath, amsfonts, amssymb}
\usetheme[height=8mm]{Rochester}
\usecolortheme[RGB={255,150,0}]{structure}
\usepackage{fancybox}
\setbeamertemplate{navigation symbols}{}
\usepackage{graphicx}
\usepackage[brazilian]{babel}
\usepackage{helvet}
\usepackage{ragged2e}
\beamertemplatenavigationsymbolsempty
\renewcommand{\familydefault}{\sfdefault}
\renewcommand{\l}{\longrightarrow}
\title{Convergência de Operadores e o Teorema de Toeplitz}
\author{Alexandre do Amaral \and João Vitor Parada Poletto\\ \small{Professor: José Carlos Corrêa Eidam}}
\date{20 de novembro de 2019}
\institute{Universidade Federal do Paraná}
\begin{document}
\frame{\titlepage \begin{figure}[br]
\includegraphics[scale=0.12]{logo-dmat.png}
\end{figure}}

\frame{\frametitle{Definições}

\begin{block}{Convergência de sequências de operadores}
\justifying
Seja $X$ e $Y$ espaços normados. Uma sequência de operadores $(T_n)$ de operadores $T_n \in B(X,Y)$ é dita:
\justifying
\begin{enumerate}
	\item[(1)]\textbf{uniformemente operador convergente} se existe um operador $T: X \l Y$ tal que $\Vert T_n - T \Vert \l 0$. \pause
	\item[(2)]\textbf{fortemente operador convergente} se existe um operador $T: X \l Y$ tal que $\Vert T_nx - Tx \Vert \l 0$ para todo $x \in X$. \pause
	\item[(3)] \textbf{fracamente operador convergente} se existe um operador $T: X \l Y$ tal que $\vert f(T_nx) - (Tx) \vert \l 0$ para todo $x \in X$ e para todo $f \in Y'$. \pause \linebreak \\
	
	$T$ é chamado o operador limite \textit{uniforme, forte e fraco} de $T_n$, respectivamente.
	\end{enumerate}

\end{block}}

\frame{\frametitle{Exemplos}

No espaço $l^2$ nós consideramos a sequência $(T_n)$, onde $T_n: l^2 \l l^2$ é definido por:

		$$ T_nx = (\underbrace{0,0,...,0}_{n zeros},\xi_{n+1}, \xi_{n+2}, \xi_{n+3},...);$$
		
com $x = (\xi_1, \xi_2,...) \in l^2$. Este operador é linear e limitado e é fortemente convergente para 0 mas não é uniformemente convergente.}

\frame{\frametitle{Exemplos}

 Ainda no espaço $l^2$, uma outra sequência $T_n$ de operadores $T_n: l^2 \l l^2$ é definida por:

		$$ T_nx = (\underbrace{0,0,...,0}_{n zeros},\xi_{1}, \xi_{2}, \xi_{3},...);$$
		
	com $x = (\xi_1, \xi_2,...) \in l^2$. O operador $T_n$ é linear e limitado e mostraremos que ele é fracmente operador convergente para $0$ mas não fortemente. Para isso recorramos ao Teorema de Riesz's.}
	
\frame{\frametitle{Teorema}

\begin{block}{Teorema de Riesz's (Funcionais no espaço de Hilbert)}
\justifying
	Todo funcional linear $f$ em um espaço de Hilbert pode ser representado em termos do produto interno, chamado:
	
	$$f(x) = {\langle {x},{z}\rangle}$$
	
onde $z$ depende de $f$, é unicamente determinado pelo funcional e tem norma:

$$ \Vert z \Vert = \Vert f \Vert $$

\end{block}}

\frame{\frametitle{Exemplos}

 Qualquer funcional $f \in l^2$ pode ser escrito como 
 
 $$f(x) = {\langle {x},{z}\rangle} = \sum_{j = 1}^{\infty} \xi_j \overline{\eta}_j $$
 
 onde $z = (\eta_j) \in l^2$. Desta forma, chamando $ J = n + k $
 
  $$f(T_nx) = \langle {T_nx},{z}\rangle = \sum_{j = n + 1}^{\infty} \xi_{j - n} \overline{\eta}_j = \sum_{k = 1}^{\infty} \xi_{k} \overline{\eta}_{n + k} $$}
	
\end{document}

\end{document}