\documentclass[xcolor=dvipsnames]{beamer}
\usepackage[utf8]{inputenc}
\usepackage{amsmath, amsfonts, amssymb}
\usetheme[height=8mm]{Rochester}
\usecolortheme[RGB={255,150,0}]{structure}
\usepackage{fancybox}
\setbeamertemplate{navigation symbols}{}
\usepackage{graphicx}
\usepackage[brazilian]{babel}
\usepackage{helvet}
\usepackage{ragged2e}
\beamertemplatenavigationsymbolsempty
\renewcommand{\familydefault}{\sfdefault}
\renewcommand{\l}{\longrightarrow}
\title{Convergência de Operadores e o Teorema de Toeplitz}
\author{Alexandre do Amaral \and João Vitor Parada Poletto\\ \small{Professor: José Carlos Corrêa Eidam}}
\date{20 de novembro de 2019}
\institute{Universidade Federal do Paraná}
\begin{document}
\frame{\titlepage \begin{figure}[br]
\includegraphics[scale=0.12]{logo-dmat.png}
\end{figure}}

\frame{\frametitle{Definições}

\begin{block}{Convergência de sequências de operadores}
\justifying
Seja $X$ e $Y$ espaços normados. Uma sequência de operadores $(T_n)$ de operadores $T_n \in B(X,Y)$ é dita:
\justifying
\begin{enumerate}
	\item[(1)]\textbf{uniformemente operador convergente} se existe um operador $T : X \l Y$ tal que $\Vert T_n - T \Vert \l 0$. \pause
	\item[(2)]\textbf{fortemente operador convergente} se existe um operador $T: X \l Y$ tal que $\Vert T_nx - Tx \Vert \l 0$ para todo $x \in X$. \pause
	\item[(3)] \textbf{fracamente operador convergente} se existe um operador $T: X \l Y$ tal que $\vert f(T_nx) - f(Tx) \vert \l 0$ para todo $x \in X$ e para todo $f \in Y'$. \pause \linebreak \\
	
	$T$ é chamado o operador limite \textit{uniforme, forte e fraco} de $T_n$, respectivamente.
	\end{enumerate}

\end{block}}

\frame{\frametitle{Exemplos}
\justifying
No espaço $l^2$ nós consideramos a sequência $(T_n)$, onde $T_n: l^2 \l l^2$ é definido por:

		$$ T_nx = (\underbrace{0,0,\cdots,0}_{n \, zeros},\xi_{n+1}, \xi_{n+2}, \xi_{n+3},\cdots);$$
		
com $x = (\xi_1, \xi_2,\cdots) \in l^2$.\pause Este operador é linear e limitado e é fortemente convergente para 0 mas não é uniformemente convergente.}

\frame{\frametitle{Exemplos}
\justifying
 Ainda no espaço $l^2$, uma outra sequência $T_n$ de operadores $T_n: l^2 \l l^2$ é definida por:

		$$ T_nx = (\underbrace{0,0,\cdots,0}_{n \, zeros},\xi_{1}, \xi_{2}, \xi_{3},\cdots);$$
		
	com $x = (\xi_1, \xi_2,\cdots) \in l^2$. \pause O operador $T_n$ é linear e limitado e mostraremos que ele é fracmente operador convergente para $0$ mas não fortemente. Para isso recorramos ao Teorema de Riesz.}
	
\frame{\frametitle{Teorema}
\justifying
\begin{block}{Teorema de Riesz (Funcionais no espaço de Hilbert)}
\justifying
	Todo funcional linear $f$ em um espaço de Hilbert pode ser representado em termos do produto interno, chamado:
	
	$$f(x) = {\langle {x},{z}\rangle}$$
	
onde $z$ depende de $f$, é unicamente determinado pelo funcional e tem norma:

$$ \Vert z \Vert = \Vert f \Vert $$

\end{block}}

\frame{\frametitle{Exemplos}
\justifying
 Qualquer funcional $f \in l^2$ pode ser escrito como 
 
 $$f(x) = {\langle {x},{z}\rangle} = \sum_{j = 1}^{\infty} \xi_j \overline{\eta}_j $$
 
 onde $z = (\eta_j) \in l^2$. \pause Desta forma, chamando $ j = n + k $
 
  $$f(T_nx) = \langle {T_nx},{z}\rangle = \sum_{j = n + 1}^{\infty} \xi_{j - n} \overline{\eta}_j = \sum_{k = 1}^{\infty} \xi_{k} \overline{\eta}_{n + k} $$ \pause
  Usando a desigualdade de Holder para somas, temos que:
  
  $$\vert f(T_nx) \vert = \vert \sum_{k = 1}^{\infty} \xi_{k} \overline{\eta}_{n + k} \vert \leq \sum_{k = 1}^{\infty} \vert \xi_{k} \overline{\eta}_{n + k} \vert \leq $$
  
  $$ {\left({\sum_{k = 1}^{\infty} {\vert \xi_{k} \vert}^2}\right)}^{1/2}  {\left({\sum_{m = n +1}^{\infty} {\vert \xi_{m} \vert}^2}\right)}^{1/2} $$
   }
   
\frame{\frametitle{Exemplos}
\justifying
Elevando ambos os lados ao quadrado:
 $$ {\vert f(T_nx) \vert} = {\left({\sum_{k = 1}^{\infty} {\vert \xi_{k} \vert}^2}\right)}  {\left({\sum_{m = n +1}^{\infty} {\vert \xi_{m} \vert}^2}\right)} $$
Como o lado direito da desigualdade vai a $0$, temos que $f(T_nx) \l 0$. Consequentemente, $(T_n)$ é fracamamente operador convergente para $0$. \pause

Porém $T_nx$ não é fortemente operador convergente, basta considerarmos a sequência $x = (1,0,0,\cdots)$
}

\frame{\frametitle{Convergência de funcionais}
\justifying
E sobre convergência de funcionais? \\ \pause

Como funcionais são operadores lineares, as definições citadas anterioremente se aplicam. Além disso, para uma sequência de funcionais, convergência operador forte e fraca são equivalentes, basta relembrar o seguinte teorema: 
}

\frame{\frametitle{Teorema}
\justifying
\begin{block}{Teorema (convergência forte e fraca)}
Seja $(x_n)$ uma sequência em um espaço normado de dimensão $X$. Então
\begin{enumerate}
\item[a)] Convergência forte implica convergência fraca com o mesmo limite.
\item[b)] A recíproca de \textit{a} não é sempre verdadeira.
\item[c)] \textbf{Se dim $X \leq \infty $, convergência fraca implica convergência forte.}
 \end{enumerate}

\end{block}
}

\frame{\frametitle{Convergência de funcionais}
\justifying
Para uma sequência de funcionais $f_n$ temos que $f_nx \in F$ para todo $ x \in X$. \pause

Então, assumindo que exista um funcional $f$ para o qual a sequência de funcionais converge fracamente significa que  para cada  $x \in X$ e $g \in F'$, temos que:
 $\Vert f_n - f \Vert \l 0 \iff \vert g(f_nx) - g(fx) \vert$ pelo item c do teorema anterior. \\
 
\pause Por esse motivo, há as seguintes definições:
 
 }
 
\frame{\frametitle{Convergência de funcionais}
\justifying
\begin{block}{Definção (forte e fraca* convergência de uma sequência de funcionais)} 
Seja $f_n$ uma sequência de funcionais lineares limitados no espaço normado $X$. Então:
\begin{enumerate}
\item[(a)] Convergência forte de $f_n$ significa que existe  um $f \in X'$ tal que $ \Vert f_n - f \Vert \l 0$. Escrevemos: 
	$$f_n \l f.$$ \pause
\item[(b)] Convergência fraca* de $f_n$ significa que existe um $f \in X'$ tal que $f_n(x) \l f(x)$ para todo $x \in X$. Escrevemos: \[f_n\stackrel{\scriptscriptstyle w^*}{\longrightarrow} f\]\pause
$f$ em \textit{(a)} e \textit{(b)} é chamado limite forte e limie fraco* de $f_n$, respectivamente.
\end{enumerate}

\end{block}
	}
	
\frame{\frametitle{Convergência de operadores}
\justifying
Voltando para convergência de operadores $T_n \in B(X,Y)$ o que pode ser dito do operador limite? \\ \pause

Em primeiro lugar, convergência uniforme implica que $T: X \l Y$ nas definições anteriores é limitado. \\ \pause

Em segundo lugar, se a convergência é forte ou fraca o operador ainda é linear, mas não necessariamente limitado.
}

\frame{\frametitle{Exemplo}
\justifying
O espaço $X$ das sequências $x = (\xi_j)$ no $l^2$ com somente finitos termos não nulos, na métrica do $l^2$ não é completo. \pause

É possível construir uma sequência de operadores nesse espaço que converge fortemente para um operador ilimitado. \pause

Tal sequência é: $$T_nx = (\xi_1, 2\xi_2, 3\xi_3,\cdots, n\xi_n,\xi_{n+1},\xi_{n+2},\cdots).$$
 \pause

Porém, se $X$ é completo, convergência operador forte implica que o limite dos operadores é um operador limitado. 
}

\frame{\frametitle{Lema}
\justifying
\begin{block}{Lema (Convergência forte de operadores}
Seja $T_n \in B(X,Y)$ uma sequência de operadores, onde $X$ é um espaço de Banach e $Y$ é um espaço normado. Se $T_n$ é fortemente operador convergente com limite $T$, então $T \in B(X,Y).$ \pause

\textbf{Demonstração:} A linearidade de $T$ segue direto da linearidade de $T_n$. \pause

Como  $T_nx \l Tx$ para cada $x \in X$, segue que a sequência $(T_nx)$ é limitada para cada $x$, visto que toda sequência convergente é limitada.



\end{block}
}

\frame{\frametitle{Lema}
\justifying
\begin{block}{Lema (Convergência forte de operadores}
Como $X$ é completo, $(\Vert T_n \Vert)$ é limitado pelo teorema da limitação uniforme. Digamos então que $(\Vert T_n \Vert) \leq c$ para todo $n$. Daí segue o resultado anunciado. $\blacksquare$



\end{block}
}
\end{document}