\documentclass[xcolor=dvipsnames]{beamer}
\usepackage[utf8]{inputenc}
\usepackage{amsmath, amsfonts, amssymb}
\usetheme[height=8mm]{Rochester}
\usecolortheme[RGB={255,150,0}]{structure}
\usepackage{fancybox}
\setbeamertemplate{navigation symbols}{}
\usepackage{graphicx}
\usepackage[brazilian]{babel}
\usepackage{helvet}
\usepackage{ragged2e}
\beamertemplatenavigationsymbolsempty
\renewcommand{\familydefault}{\sfdefault}
\renewcommand{\l}{\longrightarrow}
\title{Convergência de Operadores e o Teorema de Toeplitz}
\author{Alexandre do Amaral \and João Vitor Parada Poletto\\ \small{Professor: José Carlos Corrêa Eidam}}
\date{20 de novembro de 2019}
\institute{Universidade Federal do Paraná}
\begin{document}
\frame{\titlepage \begin{figure}[br]
\includegraphics[scale=0.12]{logo-dmat.png}
\end{figure}}

\frame{\frametitle{Definições}

\begin{block}{Convergência de sequências de operadores}
\justifying
Seja $X$ e $Y$ espaços normados. Uma sequência de operadores $(T_n)$ de operadores $T_n \in B(X,Y)$ é dita:
\justifying
\begin{enumerate}
	\item[(1)]\textbf{uniformemente convergente} se existe um operador $T : X \l Y$ tal que $\Vert T_n - T \Vert \l 0$. \pause
	\item[(2)]\textbf{fortemente convergente} se existe um operador $T: X \l Y$ tal que $\Vert T_nx - Tx \Vert \l 0$ para todo $x \in X$. \pause
	\item[(3)] \textbf{fracamente convergente} se existe um operador $T: X \l Y$ tal que $\vert f(T_nx) - f(Tx) \vert \l 0$ para todo $x \in X$ e para todo $f \in Y'$. \pause \linebreak \\
	
	$T$ é chamado o operador limite \textit{uniforme, forte e fraco} de $T_n$, respectivamente.
	\end{enumerate}

\end{block}}

\frame{\frametitle{Exemplos}
\justifying
No espaço $l^2$ nós consideramos a sequência $(T_n)$, onde $T_n: l^2 \l l^2$ é definido por:

		$$ T_nx = (\underbrace{0,0,\cdots,0}_{n \, zeros},\xi_{n+1}, \xi_{n+2}, \xi_{n+3},\cdots);$$
		
com $x = (\xi_1, \xi_2,\cdots) \in l^2$.\pause Este operador é linear e limitado e é fortemente convergente para 0 mas não é uniformemente convergente.}

\frame{\frametitle{Exemplos}
\justifying
 Ainda no espaço $l^2$, uma outra sequência $T_n$ de operadores $T_n: l^2 \l l^2$ é definida por:

		$$ T_nx = (\underbrace{0,0,\cdots,0}_{n \, zeros},\xi_{1}, \xi_{2}, \xi_{3},\cdots);$$
		
	com $x = (\xi_1, \xi_2,\cdots) \in l^2$. \pause O operador $T_n$ é linear e limitado e mostraremos que ele é fracamente convergente para $0$ mas não fortemente. Para isso recorramos ao Teorema de Riesz.}
	
\frame{\frametitle{Teorema}
\justifying
\begin{block}{Teorema de Riesz (Funcionais no espaço de Hilbert)}
\justifying
	Todo funcional linear $f$ em um espaço de Hilbert pode ser representado em termos do produto interno, chamado:
	
	$$f(x) = {\langle {x},{z}\rangle}$$
	
onde $z$ depende de $f$, é unicamente determinado pelo funcional e tem norma:

$$ \Vert z \Vert = \Vert f \Vert $$

\end{block}}

\frame{\frametitle{Exemplos}
\justifying
 Qualquer funcional $f \in l^2$ pode ser escrito como 
 
 $$f(x) = {\langle {x},{z}\rangle} = \sum_{j = 1}^{\infty} \xi_j \overline{\eta}_j $$
 
 onde $z = (\eta_j) \in l^2$. \pause Desta forma, chamando $ j = n + k $
 
  $$f(T_nx) = \langle {T_nx},{z}\rangle = \sum_{j = n + 1}^{\infty} \xi_{j - n} \overline{\eta}_j = \sum_{k = 1}^{\infty} \xi_{k} \overline{\eta}_{n + k} $$ \pause
  Usando a desigualdade de Holder para somas, temos que:
  
  $$\vert f(T_nx) \vert = \vert \sum_{k = 1}^{\infty} \xi_{k} \overline{\eta}_{n + k} \vert \leq \sum_{k = 1}^{\infty} \vert \xi_{k} \overline{\eta}_{n + k} \vert \leq $$
  
  $$ {\left({\sum_{k = 1}^{\infty} {\vert \xi_{k} \vert}^2}\right)}^{1/2}  {\left({\sum_{m = n +1}^{\infty} {\vert \eta_{m} \vert}^2}\right)}^{1/2} $$
   }
   
\frame{\frametitle{Exemplos}
\justifying
Elevando ambos os lados ao quadrado:
 $$ {\vert f(T_nx) \vert} = {\left({\sum_{k = 1}^{\infty} {\vert \xi_{k} \vert}^2}\right)}  {\left({\sum_{m = n +1}^{\infty} {\vert \xi_{m} \vert}^2}\right)} $$
Como o lado direito da desigualdade vai a $0$, temos que $f(T_nx) \l 0$. Consequentemente, $(T_n)$ é fracamamente operador convergente para $0$. \pause

Porém $T_nx$ não é fortemente operador convergente, basta considerarmos a sequência $x = (1,0,0,\cdots)$
}

\frame{\frametitle{Convergência de funcionais}
\justifying
E sobre convergência de funcionais? \\ \pause

Como funcionais são operadores lineares, as definições citadas anterioremente se aplicam. Além disso, para uma sequência de funcionais, convergência forte e fraca são equivalentes, basta relembrar o seguinte teorema: 
}

\frame{\frametitle{Teorema}
\justifying
\begin{block}{Teorema (convergência forte e fraca)}
Seja $(x_n)$ uma sequência em um espaço normado de dimensão $X$. Então
\begin{enumerate}
\item[a)] Convergência forte implica convergência fraca com o mesmo limite.
\item[b)] A recíproca de \textit{a} não é sempre verdadeira.
\item[c)] \textbf{Se dim $X < \infty $, convergência fraca implica convergência forte.}
 \end{enumerate}

\end{block}
}

\frame{\frametitle{Convergência de funcionais}
\justifying
Para uma sequência de funcionais $f_n$ temos que $f_nx \in F$ para todo $ x \in X$. \pause

Então, assumindo que exista um funcional $f$ para o qual a sequência de funcionais converge fracamente significa que  para cada  $x \in X$ e $g \in F'$, temos que:
 $\Vert f_n - f \Vert \l 0 \iff \vert g(f_nx) - g(fx) \vert$ pelo item c do teorema anterior. \\
 
\pause Por esse motivo, há as seguintes definições:
 
 }
 
\frame{\frametitle{Convergência de funcionais}
\justifying
\begin{block}{Definção (forte e fraca* convergência de uma sequência de funcionais)} 
Seja $f_n$ uma sequência de funcionais lineares limitados no espaço normado $X$. Então:
\begin{enumerate}
\item[(a)] Convergência forte de $f_n$ significa que existe  um $f \in X'$ tal que $ \Vert f_n - f \Vert \l 0$. Escrevemos: 
	$$f_n \l f.$$ \pause
\item[(b)] Convergência fraca* de $f_n$ significa que existe um $f \in X'$ tal que $f_n(x) \l f(x)$ para todo $x \in X$. Escrevemos: \[f_n\stackrel{\scriptscriptstyle w^*}{\longrightarrow} f\]\pause
$f$ em \textit{(a)} e \textit{(b)} é chamado limite forte e limie fraco* de $f_n$, respectivamente.
\end{enumerate}

\end{block}
	}
	
\frame{\frametitle{Convergência de operadores}
\justifying
Voltando para convergência de operadores $T_n \in B(X,Y)$ o que pode ser dito do operador limite? \\ \pause

Em primeiro lugar, convergência uniforme implica que $T: X \l Y$ nas definições anteriores é limitado. \\ \pause

Em segundo lugar, se a convergência é forte ou fraca o operador ainda é linear, mas não necessariamente limitado.
}

\frame{\frametitle{Exemplo}
\justifying
O espaço $X$ das sequências $x = (\xi_j)$ no $l^2$ com somente finitos termos não nulos, na métrica do $l^2$ não é completo. \pause

É possível construir uma sequência de operadores nesse espaço que converge fortemente para um operador ilimitado. \pause

Tal sequência é: $$T_nx = (\xi_1, 2\xi_2, 3\xi_3,\cdots, n\xi_n,\xi_{n+1},\xi_{n+2},\cdots).$$
 \pause

Porém, se $X$ é completo, convergência operador forte implica que o limite dos operadores é um operador limitado. 
}

\frame{\frametitle{Lema}
\justifying
\begin{block}{Lema (Convergência forte de operadores}
Seja $T_n \in B(X,Y)$ uma sequência de operadores, onde $X$ é um espaço de Banach e $Y$ é um espaço normado. Se $T_n$ é fortemente operador convergente com limite $T$, então $T \in B(X,Y).$ \pause

\textbf{Demonstração:} A linearidade de $T$ segue direto da linearidade de $T_n$. \pause

Como  $T_nx \l Tx$ para cada $x \in X$, segue que a sequência $(T_nx)$ é limitada para cada $x$, visto que toda sequência convergente é limitada.



\end{block}
}

\frame{\frametitle{Lema}
\begin{block}{Lema (Convergência forte de operadores)}
\justifying
Como $X$ é completo, $(\Vert T_n \Vert)$ é limitado pelo teorema da limitação uniforme. Digamos então que $\Vert T_n \Vert \leq c$ para todo $n$. Então, $\Vert T_nx \Vert \leq \Vert Tn \Vert \Vert x \Vert \leq c \Vert x \Vert .$ Daí segue o resultado anunciado. $\blacksquare$



\end{block}
}

\frame{\frametitle{Teorema}

\begin{block}{Convergência forte}
	Uma sequência $(T_n)$ de operadores $T_n \in B(X,Y)$, onde $X$ e $Y$ são espaços de Banach, é fortemente convergente se, e somente se, são satisfeitos:
\begin{enumerate} \pause
\item[(A)] A sequência $(\Vert T_n \Vert )$ é limitada. \pause
\item[(B)] A sequência $(T_nx)$ é Cauchy em $Y$ para todo $x$ em um conjunto  maximal $M$ de $X$.
\end{enumerate}

\end{block}

}

\frame{\frametitle{Teorema}

\justifying
\textbf{Demonstração:} Se $T_nx \l Tx$ para cada $x \in X$, então $(A)$ segue do lema anterior e $(B)$ é verificado facilmente. \pause

	Reciprocamente, assumindo que $(A)$ e $(B)$ são satisfeitos existe $c$ tal que $\Vert T_n \Vert \leq c$ para todo $n$. \pause Seja agora $x \in X$ e $\epsilon > 0$. Como $M$ é maximal em $X$ segue que span $M$ é denso em $X$. Então existe $y \in span M$ tal que $$\Vert x - y \Vert < \dfrac{\epsilon}{3c}.$$

}

\frame{\frametitle{Teorema}
Como $y \in span M$, a sequência $(T_ny)$ é Cauchy em $(B)$ Consequentemente, existe um $N$ tal que $$\Vert T_ny - T_my \Vert \leq \dfrac{\epsilon}{3}.$$ para $m,n > N$. \pause

Utilizando as desigualdades triangulares temos o resultado provado. $\blacksquare$
}

\frame{\frametitle{Aplicações}

Considerando uma sequência de funconais $(f_n)$ no teorema anterior obtemos aplicações interessantes, como veremos a seguir.
}





\frame{\frametitle{Métodos de Somabilidade}
  Com intuito de generalizar a noção de convergência de sequências podemos utilizar métodos de somabilidade, que associam uma sequência com outra possivelmente convergente.
  \begin{block}{Métodos matriciais}
    Um método de somabilidade é nomeado matricial se é possível escreve-lo na forma:
    \begin{equation*}
      y=Ax
    \end{equation*}
    Onde $y$ e $x$ são vetores coluna infinitos e $A$ é uma matriz infinita.
  \end{block}
}

\frame{\frametitle{Método de Cesàro}
  \begin{block}{Definição}
    O método de Cesàro é a sequência de médias até o n-ésimo termo
  \end{block}
  \begin{block}{Representação em somatório}
    Dada uma sequência $x=(\xi_n)_n$ temos que sua sequência de Cesáro $y=(\eta_n)_n$ é tal que
    \begin{equation*}
      \eta_n=\sum_{k=1}^{\infty}\alpha_{nk}\xi_k
    \end{equation*}
    \begin{equation*}
      \alpha_{nk}=
      \begin{cases}
        \frac{1}{n} &\quad\text{se}\quad k\leq n \\
        0 &\quad\text{se}\quad k>n
      \end{cases}
    \end{equation*}
  \end{block}
}

\frame{\frametitle{Representação matricial do método de Cesàro}
  \begin{equation*}
  {\Large
    A=
    \begingroup
    \renewcommand*{\arraystretch}{1.5}
    \begin{bmatrix}
      \frac{1}{1} & 0                 & 0 & 0 & 0 & \\
      \frac{1}{2} & \frac{1}{2} & 0                 & 0 & 0 & \dots\\
      \frac{1}{3} & \frac{1}{3} & \frac{1}{3} & 0                 & 0 & \\
      \frac{1}{4} & \frac{1}{4} & \frac{1}{4} & \frac{1}{4} & 0 & \\
                         & \vdots         &                     &                    & \ddots &
    \end{bmatrix}
    \endgroup
  }
  \end{equation*}
}

\frame{\frametitle{Regularidade e notação}
   Um método matricial pode ser chamado de A-método, se todas as linhas da matriz infinita e $y=(\eta_n)_n$ convergem no sentido usual. O limite é denominado A-limite de $x$ e $x$ é dito A-somável, o conjunto de todas as sequências somáveis é denominado A-domínio.
   \begin{block}{Regularidade}
     Um método matricial é dito regular se toda sequência convergente pertence ao seu domínio e seu A-limite coincide com o limite usual da sequência.
   \end{block}
}

\frame{\frametitle{Teorema de Toeplitz}
  Um método matricial é regular se e somente se,
  \begin{align}
    &\lim_{n\to\infty}\alpha_{nk}=0 \label{lim0}\\
    &\lim_{n\to\infty}\sum_{k=1}^{\infty}\alpha_{nk}=1 \label{limsum1}\\
    &\sum_{k=1}^{\infty}\left\lvert\alpha_{nk}\right\rvert \leq \gamma \label{sumgamma}
  \end{align}
}

\frame{\frametitle{Prova}
  \begin{block}{Regularidade$\implies$ Equação~\ref{lim0}}
    Dada a sequência $x_k$ com todo termo igual a $0$, exceto o termo $k$ que é igual a $1$, temos que $\eta_n=\alpha_{nk}$
  \end{block}
  \begin{block}{Regularidade$\implies$ Equação~\ref{limsum1}}
    Dada a sequência $x$ com todo termo igual a $1$ temos que
    \begin{equation*}
      \eta_n=\sum_{k=1}^{\infty}\alpha_{nk}
    \end{equation*}
  \end{block}
}

\frame{\frametitle{Prova}
  \begin{block}{Regularidade$\implies$ Equação~\ref{sumgamma}}
    Dado o espaço metrico $c$ com a norma $\ell^{\infty}$ \\
    definimos funcionais lineares
    \begin{equation*}
      f_{nm}(x)=\sum_{k=1}^m\alpha_{nk}\xi_k \quad m\text{, }n = 1\text{,}2\text{,}\dots
    \end{equation*}
    então temos que o método define funcionais lineares
    \begin{equation*}
      \eta_n=f_n(x)=\sum_{k=1}^{\infty}\alpha_{nk}\xi_k \quad n=1\text{,}2\text{,}\dots
    \end{equation*}
    \begin{equation*}
      f_{nm}(x)\to f_n(x) \quad \text{ou seja} \quad f_{nm}\overset{w^\ast}{\to}f_n
    \end{equation*}
  \end{block}
}

\frame{\frametitle{Prova}
  \begin{block}{Regularidade$\implies$ Equação~\ref{sumgamma}}
    \begin{equation*}
      \xi_k^{(n,m)}=
      \begin{cases}
        \frac{\alpha_{nk}}{\left\lvert\alpha_{nk}\right\rvert} &\quad \text{se} \quad k\leq m \quad \text{e} \quad \alpha_{nk}\neq 0 \\
        0 &\quad \text{se} \quad k>m \quad \text{ou} \quad \alpha_{nk}=0
      \end{cases}
    \end{equation*}
    então temos
    \begin{equation*}
      \sum_{k=1}^m\left\lvert\alpha_{nk}\right\rvert=f_{nm}(x_{nm})\leq\left\lVert f_{nm}\right\rVert
    \end{equation*}
    \begin{equation*}
      \sum_{k=1}^{\infty}\left\lvert\alpha_{nk}\right\rvert\leq\left\lVert f_n\right\rVert\leq\gamma
    \end{equation*}
  \end{block}
}

\frame{\frametitle{Prova}
  \begin{block}{Equações~\ref{lim0},~\ref{limsum1} e~\ref{sumgamma} $\implies$ Regularidade}
    Definimos o funcional linear
    \begin{equation*}
      f(x)=\xi=\lim_{k\to\infty}\xi_k
    \end{equation*}
    que é limitado pois
    \begin{equation*}
	      \left\lvert f(x)\right\rvert\leq\sup_{k\in\mathbb{N}}\left\lvert\xi_k\right\rvert=\left\lVert x\right\rVert
    \end{equation*}
    Dado o conjunto $M$ das sequências quase constantes, que é denso em $c$, e $x\in M$ com todo termo igual apos $j$
  \end{block}
}

\frame{\frametitle{Prova}
  \begin{block}{Equações~\ref{lim0},~\ref{limsum1} e~\ref{sumgamma} $\implies$ Regularidade}
    \begin{align*}
      \eta_n=f_n(x)&=\sum_{k=1}^{j-1}\alpha_{nk}\xi_k+\xi\sum_{k=j}^{\infty}\alpha_{nk} \\
                             &=\sum_{k=1}^{j-1}\alpha_{nk}(\xi_k-\xi)+\xi\sum_{k=1}^{\infty}\alpha_{nk} \\
                             &\text{Pelas Equações~\ref{lim0} e~\ref{limsum1}} \\
      \eta_n=f_n(x)&\to 0+\xi*1=\xi=f(x)
    \end{align*}
  \end{block}
}

\frame{\frametitle{Prova}
  \begin{block}{Equações~\ref{lim0},~\ref{limsum1} e~\ref{sumgamma} $\implies$ Regularidade}
    E pela Equação~\ref{sumgamma} temos
    \begin{equation*}
      \left\lvert f_n(x)\right\rvert\leq\left\lVert x\right\rVert\sum_{k=1}^{\infty}\left\lvert\alpha_{nk}\right\rvert\leq\gamma\left\lVert x\right\rVert
    \end{equation*}
    então $f_n$ é limitado, e como $f_n(x)\to f(x),x\in M$ logo $f_n\overset{w^\ast}{\to}f$. \\
    Portanto se $f(x)$ existe $\eta_n\to\xi$ e portanto é regular.
  \end{block}
}

\frame{\frametitle{Exemplos}
  \begin{block}{Inverso do método de Cesàro}
    Podemos um inverso para o método de Cesàro pois, nota-se que
    \begin{align*}
      \xi_1&=\eta_1 \\
      \xi_2&=2\eta_2-\eta_1 \\
      \xi_3&=3\eta_3-2\eta_2 \\
      \\
      \xi_n&=n\eta_n-(n-1)\eta_{n-1}
    \end{align*}
  \end{block}
}

\frame{\frametitle{Exemplos}
  \begin{block}{Matriz do inverso do método de Cesàro}
    {\Large
    \begin{equation*}
      \begin{bmatrix}
        1  & 0  & 0 & 0 & 0 & \\
        -1 & 2  & 0 & 0 & 0 & \dots\\
        0  & -2 & 3 & 0 & 0 & \\
        0  & 0 & -3 & 4 & 0 & \\
            & \vdots & & & \ddots &
      \end{bmatrix}
    \end{equation*}
    }
  \end{block}
}

\end{document}
